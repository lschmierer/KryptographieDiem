\documentclass{scrartcl}
\usepackage{selinput}
\SelectInputMappings{
  adieresis={ä},
  germandbls={ß},
}
\usepackage[ngerman]{babel}


\usepackage{csquotes}
\MakeOuterQuote{"}
\usepackage{biblatex}

\usepackage{amsmath}
\usepackage{amssymb}

\usepackage{dirtree}

\bibliography{quellen}

\begin{document}

\titlehead{
  Universität Leipzig \\
  Fakultät für Mathematik und Informatik \\
  Institut für Informatik
}
\subject{Projekt Dokumentation}
\title{Generische Methoden für das Faktorisieren und die Berechnung diskreter Logarithmen}
%\subtitle{}
\author{Schindler, Benjamin \and Schmierer, Lukas}
\date{\today}
\publishers{Dr. Claus Diem}
\maketitle

\tableofcontents

\section{Projekt Organisation}
\label{sec:organisation}

\subsection{Verzeichnisstruktur}
\label{sec:verzeichnisstruktur}
Eine Schematische Darstellung der Verzeichnisstruktur ist in Abbildung \ref{fig:verzeichnisstruktur} zu sehen.\\
In unserem Projekt befinden sich drei Verzeichnisse:

\begin{itemize}
\item \emph{tocas} - Enthält den von Dr. Claus Diem bereitgestellten Python Code aus \cite{tocas} unverändert.

\item \emph{ha} - Abkürzung für \emph{Hausaufgaben}. Enthält unsere Python Implementierung zur Lösung der Hausaufgaben aus der Vorlesung "Mathematische Grundlagen der Kryptographie mit öffentlichen Schlüsseln"
\item \emph{projekt} - Enthält den im Rahmen unserer Projektarbeit geschriebenen Python Code zur Lösung des Diskreten Logarithmus Problems und zur Faktorisierung mithilfe generischer Methoden.
\end{itemize}
Da das \emph{tocas}-Projekt separat vorliegt, können dadurch zukünftige Änderungen und Updates in unser Projekt eingepflegt werden. Außerdem erreichen wir so eine strikte Trennung von bereitgestelltem und eigenem Code.\\
Wenn zu bereits existierenden Klassen aus \emph{tocas} weitere Funktionalitäten hinzugefügt wurden, wurden diese in externe Module geschrieben, die den Datei-Suffix \emph{\_extension} tragen. So verschönert zum Beispiel das Modul \emph{'format\_extension.py'} die Standard-Ausgabe von Tocas (z.B. die Ausgabe für den Ring der ganzen Zahlen mit \( \mathbb{Z} \) statt Z). \\
Innerhalb der Ordner \emph{ha} und \emph{project} befindet sich jeweils ein Unterordner \emph{tests}. In diesem sind verschiedene Tests geschrieben, die mit der Web-Applikation \emph{Jupyter Notebook} aus \cite{jupyterNotebook} ausgeführt werden können.

\begin{figure}[h]
  \fbox{
    \begin{minipage}{0.9\textwidth}
    \dirtree{%
    .1 projekt\_generisches\_faktorisieren\_und\_dlp.
    .2 tocas.
    .3 ....
    .2 ha.
    .3 tests.
    .4 ....
    .3 format\_extension.py.
    .3 ....
    .2 projekt.
    .3 tests.
    .4 ....
    .3 ....
    .2 dokumentation.pdf.
    }
    \end{minipage}
  }
  \caption{Verzeichnisstruktur des Projekts}
  \label{fig:verzeichnisstruktur}
\end{figure}

\subsection{Tests}
\label{sec:tests}

Zu jedem Teilgebiet unseres Projekts wurden zugehörige Tests als Jupyter Notebooks erstellt. Dabei gibt der Name des Tests an, welche Funktionalität getestet wird. Eine Liste aller Tests die wir im Rahmen unserer Projektarbeit geschrieben haben sind inklusive einer kurzen Beschreibung in Abbbildung \ref{fig:tests} zu sehen. [Anmerkung, Benny: Ich würde die Beschreibungen in der Abbildung dann halt noch jeweils mit 1-2 Sätzen erweitern, wenn du das so gut findest. Ansonsten könnten wir die Tests natürlich auch im nächsten Kapitel ausführlicher beschreiben]

\begin{figure}[h]
 \centering
 \renewcommand{\arraystretch}{1.5}
 \begin{tabular}{r|p{8cm}}
  \emph{test\_edwards\_kurven.ipynb} & Elliptische Kurven in Edwards-Darstellung  \\
  \hline \emph{test\_weierstrass\_kurven.ipynb} & Elliptische Kurven in Weierstrass-Darstellung \\
  \hline \emph{test\_pohlig\_hellman.ipynb} & Test des Pohlig-Hellman-Algorithmus zur Berechnung des diskreten Logarithmus Problems \\
  \hline \emph{test\_baby\_step\_giant\_step.ipynb} & Test des Baby-Step-Giant-Step-Algorithmus Berechnung des diskreten Logarithmus Problems \\
  \hline \emph{test\_rho.ipynb} & Test des Rho-Algorithmus in verschiedenen Varianten zur Berechnung des diskreten Logarithmus Problems \\
  \hline \emph{test\_kaenguru.ipynb} & Test des Kängeru-Algorithmus (auch Lamda-Methode genannt) zur Berechnung des diskreten Logarithmus Problems \\
 \end{tabular}
 \renewcommand{\arraystretch}{1}
 \caption{Liste aller Tests aus der Projektarbeit}
 \label{fig:tests}
\end{figure}


\section{Implementierung}
\label{sec:implementierung}

\subsection{Elliptische Kurven}
\label{sec:elliptische_kurven}

\subsubsection{Edwards Darstellung}
\label{sec:edwards_kurven}

\subsubsection{Weierstraß Darstellung}
\label{sec:weierstrass_kurven}

\subsection{Pohlig Hellman Reduktion}
\label{sec:pohlig_hellman}

modular austauschbarer Algorithmus für die Untergruppen
(als Funktionsparameter)

\cite{Pohlig1978}

\subsection{Baby Step Giant Step}
\label{sec:baby_step_giant_step}

\cite{Galbraith2012}

\subsection{Rho Methode}
\label{sec:rho}

modular walk als parameter

\cite{Galbraith2012}

\subsubsection{Walk Funktion}
\label{sec:walk_funktion}
original und additiv

\subsubsection{Floyd Cycle}
\label{sec:floyd_cycle}

% sqrt(πN/8)

\subsubsection{Brent Cycle}
\label{sec:brent_cycle}

\cite{Brent1980}

\subsubsection{Distinguished Points}
\label{sec:distinguished_points}

% sqrt(πr/2)

\cite{VanOorschot1999}

\subsection{Känguru Methode}
\label{sec:kaenguru}

\subsection{Vergleich}

hier performance vergleichen mit diagrammen

\subsection{Rho Methode für das Faktorisieren}
\label{sec:rho_faktorisieren}

\newpage
\printbibliography[heading=bibintoc]

\end{document}