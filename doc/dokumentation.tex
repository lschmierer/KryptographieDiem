\documentclass{scrartcl}
\usepackage{selinput}
\SelectInputMappings{
  adieresis={ä},
  germandbls={ß},
}
\usepackage[ngerman]{babel}

\usepackage{csquotes}
\usepackage{biblatex}

\usepackage{amsmath}
\usepackage{amssymb}

\usepackage{dirtree}

\bibliography{quellen}

\begin{document}

\titlehead{
  Universität Leipzig \\
  Fakultät für Mathematik und Informatik \\
  Institut für Informatik
}
\subject{Projekt Dokumentation}
\title{Generische Methoden für das Faktorisieren und die Berechnung diskreter Logarithmen}
%\subtitle{}
\author{Schindler, Benjamin \and Schmierer, Lukas}
\date{\today}
\publishers{Dr. Claus Diem}
\maketitle

\tableofcontents

\section{Projekt Organisation}
\label{sec:organisation}

\subsection{Verzeichnisstruktur}
\label{sec:verzeichnisstruktur}

drei Verzeichnisse

\begin{itemize}
\item \emph{tocas} enthählt den von bereitgestellte Python Code unverändert

\item \emph{ha} (Abkürzung führ \emph{Hausaufgaben}) -> Python Code für die
  Lösung der Hausaufgaben parallel zu der Vorlesungsreihe

\item \emph{projekt}
\end{itemize}

tocas seperat, da so Änderungen von Diem einfach eingepflegt werden können
Trennung von bereitgestelltem und eigenem Code

im eigenen Teil Konvention: Module mit Suffix \emph{\_extension} fügen
Funktionalität zu (in \emph{tocas} existierenden Klassen) hinzu

Beispiel \emph{format\_extension} macht Ausgabe etwas schöner (z.B. \( \mathbb{Z} \) statt Z)

ha und projekt Unteroderner tests für jupyter Notebook

siehe Abbildung \ref{fig:verzeichnisstruktur}

\begin{figure}
  \fbox{
    \begin{minipage}{0.9\textwidth}
    \dirtree{%
    .1 projekt\_generisches\_faktorisieren\_und\_dlp.
    .2 tocas.
    .3 ....
    .2 ha.
    .3 tests.
    .4 ....
    .3 format\_extension.py.
    .3 ....
    .2 projekt.
    .3 tests.
    .4 ....
    .3 ....
    .2 dokumentation.pdf.
    }
    \end{minipage}
  }
  \caption{Verzeichnisstruktur des Projekts}
  \label{fig:verzeichnisstruktur}
\end{figure}

\subsection{Tests}
\label{sec:tests}

Tests als Jupyter Notebooks

Name des Tests gibt Aufschluss über getestete Funktionalität

\section{Implementierung}
\label{sec:implementierung}

\subsection{Elliptische Kurven}
\label{sec:elliptische_kurven}

\subsubsection{Edwards Darstellung}
\label{sec:edwards_kurven}

\subsubsection{Weierstraß Darstellung}
\label{sec:weierstrass_kurven}

\subsection{Pohlig Hellman Reduktion}
\label{sec:pohlig_hellman}

modular austauschbarer Algorithmus für die Untergruppen
(als Funktionsparameter)

\cite{Pohlig1978}

\subsection{Baby Step Giant Step}
\label{sec:baby_step_giant_step}

\cite{Galbraith2012}

\subsection{Rho Methode}
\label{sec:rho}

modular walk als parameter

\cite{Galbraith2012}

\subsubsection{Walk Funktion}
\label{sec:walk_funktion}
original und additiv

\subsubsection{Floyd Cycle}
\label{sec:floyd_cycle}

% sqrt(πN/8)

\subsubsection{Brent Cycle}
\label{sec:brent_cycle}

\cite{Brent1980}

\subsubsection{Distinguished Points}
\label{sec:distinguished_points}

% sqrt(πr/2)

\cite{VanOorschot1999}

\subsection{Känguru Methode}
\label{sec:kaenguru}

\subsection{Vergleich}

hier performance vergleichen mit diagrammen

\subsection{Rho Methode für das Faktorisieren}
\label{sec:rho_faktorisieren}

\printbibliography[heading=bibintoc]

\end{document}